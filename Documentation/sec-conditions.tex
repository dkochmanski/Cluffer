\section{Conditions}

\sysname{} defines a number of conditions that are signaled when
\sysname{} is unable to fulfill the contract stipulated by the
protocol function being used.

\Defcondition {cluffer:cluffer-error}

This condition type is the base of all error conditions signaled by
\sysname{}.  Client code that wishes to handle all error conditions
signaled by \sysname{} may use this condition in its condition
handlers.

\Defcondition {cluffer:end-of-line}

This condition is signaled when an attempt is made to use a
position that is too large, either by moving a cursor there, or by
attempting to access an item in such a position.  Notice that in
some cases, "too large" means "strictly greater than the number of
items in a line", and sometimes it means "greater than or equal to
the number of items in a line".  For example, it is perfectly
acceptable to move a cursor to a position that is equal to the
number of items in a line, but it is not acceptable to attempt to
access an item in a line at that position.

\Defcondition {cluffer:beginning-of-buffer}

\Defcondition {cluffer:end-of-buffer}

\Defcondition {cluffer:cursor-attached}

This condition is signaled when an attempt is made to use a cursor
in an operation that requires that cursor to be detached, but the
cursor used in the operation is attached to a line.

\Defcondition {cluffer:cursor-detached}

This condition is signaled when an attempt is made to use a cursor
in an operation that requires that cursor to be attached, but the
cursor used in the operation is not attached to any line.

\Defcondition {cluffer:cursors-are-not-comparable}

This condition is signaled when an attempt is made to compare two
cursors which are each attached to a line but the lines do not belong
to the same buffer.  The readers \texttt{cursor1} and \texttt{cursor2}
can be used to obtain the offending cursor objects.

\Defcondition {cluffer:line-detached}

This condition is signaled when an attempt is made to use a line
in an operation that requires the line to be attached to a buffer,
but the line used in the operation is not attached to a buffer.
An example of such an operation would be to attempt to get the
line number of the line, given that the line number of a line is
determined by the buffer to which the line is attached.

\Defcondition {cluffer:object-must-be-line}

This condition is signaled by protocol generic functions that take
a line object as an argument, but something other than a line
object was given.

\Defcondition {cluffer:object-must-be-buffer}

This condition is signaled by protocol generic functions that take
a buffer object as an argument, but something other than a buffer
object was given.
